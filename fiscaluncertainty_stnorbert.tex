\documentclass{beamer}
\usepackage{beamerthemeshadow}
\usepackage{verbatim}
\usepackage{array}

\usepackage{lastpage}
\usepackage{xcolor}
\usepackage{pgf}
\usepackage{colortbl}
\usepackage{hyperref}
\usepackage{multirow}

\usepackage{siunitx}
\sisetup{input-symbols=(), group-digits  = false} 

\newcommand{\bi}{\begin{itemize}}
\newcommand{\ei}{\end{itemize}}
\newcommand{\be}{\begin{enumerate}}
\newcommand{\ee}{\end{enumerate}}
\newcommand{\bd}{\begin{description}}
\newcommand{\ed}{\end{description}}
\newcommand{\prbf}[1]{\textbf{#1}}
\newcommand{\prit}[1]{\textit{#1}}
\newcommand{\beq}{\begin{equation}}
\newcommand{\eeq}{\end{equation}}
\newcommand{\bdm}{\begin{displaymath}}
\newcommand{\edm}{\end{displaymath}}

\newcommand{\ft}[1]{
  \frametitle{\begin{tabular}{p{4.2in}r} \textcolor{white}{#1} & \small{\insertframenumber / \inserttotalframenumber} \end{tabular}}
  \setbeamercovered{transparent=18}
}

\newcommand{\eft}[1]{
  \frametitle{\begin{tabular}{p{4in}r} \textcolor{white}{#1} & \small{\hyperlink{f:questions}{\beamergotobutton{GO BACK}}} \end{tabular}}
  \setbeamercovered{transparent=18}
}

\newcommand{\stepinv}{\setbeamercovered{invisible}}
\newcommand{\stopinv}{\setbeamercovered{transparent=18}}
\newcommand{\uncoverinv}[1]
{
  \setbeamercovered{invisible}
  \uncover<+->{#1}
  \setbeamercovered{transparent=18}
}
\newcommand{\ans}[1]{\textcolor{blue}{#1}}
\newcommand{\ansinv}[1]
{
  \setbeamercovered{invisible}
  \uncover<+->{\textcolor{blue}{#1}}
  \setbeamercovered{transparent=18}
}
\newcommand{\setinv}{\setbeamercovered{invisible}}
\newcommand{\setvis}{\setbeamercovered{transparent=18}}
\newcommand{\centerpic}[2]
{
  \begin{center}
  \includegraphics[#1]{#2}
  \end{center}
}
\newcommand{\h}[1]{\hat{#1}}
\newcommand{\ds}{\displaystyle}

\definecolor{light}{rgb}{0,0.95,0.2}
\definecolor{BrickRed}{rgb}{0.05,0.4,0.05}
\newcommand{\hl}[1]{\only<#1>{\cellcolor{light}}}

\definecolor{mycolor}{rgb}{0.2,0.6,0.2}
\usecolortheme[named=mycolor]{structure}

\title[Fiscal Policy Uncertainty and Macroeconomic Consequences]{Fiscal Policy Uncertainty and Its Macroeconomic Consequences}
\author[James Murray, University of Wisconsin - La Crosse]
{
James Murray\\
Department of Economics\\
University of Wisconsin - La Crosse
}
\date{St. Norbert College\\Economics Club Seminar\\April 24, 2015}

\begin{document}

\frame{\titlepage \setcounter{framenumber}{0}}

\frame
{
  \ft{Basic Background}
  \begin{block}{What is fiscal policy?}
  \bi
  \item Government alters \textbf{fiscal variables} to influence economic outcomes.
  \item \textbf{Fiscal variable examples:} government expenditures, taxes, transfers (unemployment benefits, Medicare, SNAP).
  \ei
  \end{block}

  \pause\begin{block}{Automatic vs Discretionary Policy}
    \bi
    \item \textbf{Automatic fiscal policy:} fiscal variables are designed to move \textit{automatically} in response to macroeconomic outcomes \\
      Examples: tax collections, unemployment, Medicare, SNAP
    \item \textbf{Discretionary fiscal policy:} When elected officials identify economic problems, pass new legislation that adjusts fiscal variables.
      Examples: Unemployment benefit extension, tax rebate, government expenditure program.
    \ei
  \end{block}
}

\frame
{
  \ft{Expected fiscal behavior}
  \begin{block}{Suppose unemployment is up, production and incomes are down}
    \bi
    \item Government spending increases (discretionary to stimulate consumer spending)
    \item Tax collections should decrease.
    \item Transfers should increase.
    \item Government debt should increase.
    \ei
  \end{block}

  \pause\begin{block}{Suppose government debt to GDP is higher}
    \bi
    \item Tax collections should increase.
    \item Government expenditures should decrease.
    \item Transfers should decrease.
    \ei
  \end{block}
}

\frame
{
  \ft{Expectations Matter}
  \begin{block}{Expectations for fiscal policy matter}
    \bi
    \item Consumers' decisions influenced by expectations for future taxes, transfer benefits, government debt accumulation.
    \item Businesses' investment decisions influenced by expectations for future taxes, government spending, transfers.
    \item Degree of confidence / uncertainty in their expectations matter.
    \ei
  \end{block}

  \pause\begin{block}{Some questions}
    \bi
    \item How might economic agents figure out fiscal behavior?
    \item Can we measure how confidence they are?
    \item Has uncertainty about fiscal policy changed over time?
    \item What are the economic consequences to fiscal uncertainty?
    \ei
  \end{block}
}


\frame
{
  \ft{Purpose 1: Quantify Fiscal Uncertainty}

  \begin{block}{Present Paper}
      \bi
      \item How well do economic agents understand the behavior of fiscal policy?
      \item Every period, agents estimate behavioral equations describing fiscal policy behavior.
      \item Projection uncertainty:  Fiscal uncertainty equal to unexplained movements in fiscal policy.
      \ei
  \end{block}

  \pause\begin{block}{Existing Contributions}
    \bi
    \item DSGE (crazy mathematical) model with changing fiscal volatility:\\
      ~~~Fern\'andez-Villiverde et. al. (2011), Born and Pfeifer (2011).
    \item Index based on newspaper headlines and other real world stuff:\\
      ~~~Baker et. al. (2013)
    \ei
  \end{block}

}

\frame
{
  \ft{Purpose 2: Estimate Macroeconomic Consequences}
  \begin{block}{Fiscal Policy Variables}
    \begin{columns}
    \begin{column}{0.4\textwidth}
    \be
    \item Government Spending
    \item Tax Revenue
    \item Net Transfers
    \item Government Debt
    \ee
    \end{column}

    \begin{column}{0.5\textwidth}
        \bi
        \item \textit{Construct an uncertainty measure for each.}
        \item \textit{Construct an index for overall fiscal uncertainty}
        \ei
    \end{column}
    \end{columns}
  \end{block}

  \pause\begin{block}{Impact on Macroeconomy}
  Incorporate measures of fiscal uncertainty in ARDL models for:
    \be
    \item Consumption
    \item Investment
    \item Real GDP
    \item Employment
    \item Unemployment
    \item Inflation
    \ee  
  \end{block}
}

\frame
{
  \ft{Motivation}
  \begin{block}{Historical Economic and Political Crises}
  \bi
  \item Financial crisis and historic economic downturn.
  \item Large monetary and fiscal policy responses, fiscal policy multiplier debate is still active.
  \item U.S. Government Debt to GDP reaching historical levels.
  \item Simultaneous calls from left and right calling for opposing fiscal responses.
  \ei
  \end{block}

  \pause\begin{block}{Ben Bernanke - July 2012 Monetary Policy Report to Congress}
    \textit{``The most effective way that the Congress could help to support the economy right now would be to work to address the nation's fiscal challenges....  \textbf{\textit{Doing so earlier rather than later would help reduce uncertainty and boost household and business confidence.}}''}
  \end{block}
}

\frame
{
  \ft{Literature}
  \begin{block}{Time-varying Fiscal Volatility}
  \bi
  \item Fern\'andez-Villiverde et. al. (2011a): Fiscal policy uncertainty is stagflationary
  \item Born and Pfeifer (2011): 
    \bi
    \item Significant evidence for time-varying volatility in fiscal shocks.
    \item Not a significant driver for business cycles.
    \ei
  \item Johannsen (2012): Matters more at ZLB.
  \ei
  \end{block}

  \pause\begin{block}{Fiscal Uncertainty}
  \bi
  \item Baker (2013): Uncertainty reduces economic activity
  \item Hollmayr and Matthes (2013):
    \bi
    \item Fiscal behavior changes / evolves over time
    \item Economic agents have to learn it
    \item Permanent fiscal changes have a \textit{relatively} small impact
    \item More macroeconomic volatility
    \ei
  \ei
  \end{block}
}

\frame
{
  \ft{Forward Looking Fiscal Uncertainty}
  \begin{block}{Fiscal contractions uncertainty}
  \bi
  \item Bi, Leith, and Leeper (2013): Timing and composition of fiscal contractions
  \item Davig, Leeper, and Walker (2010):  Uncertainty re: unfunded entitlement programs is stagflationary
  \ei
  \end{block}

  \pause\begin{block}{Possibly expiring tax provisions}
  \bi
  \item Davig and Foerster (2014): Uncertainty re: expiring tax provisions decrease investment and employment
  \item Richter and Throckmorton (2014): 
    \bi
    \item Uncertainty regarding future debt target
    \item Welfare improving or reducing, depending on expectation relative to realization.
    \item Uncertainty extending the ``Bush tax cuts'' were welfare reducing.
    \ei
  \ei
  \end{block}
}

\frame
{
  \ft{Spoiler}
  \begin{block}{Consequences for Fiscal Uncertainty}
    \bi
    \item lower real GDP,
    \item lower consumption,
    \item lower investment.
    \ei
  \end{block}
 
  \pause\begin{block}{Specific fiscal variables}
    \bi
    \item Government expenditures, transfers, and debt associated with labor market contractions.
    \item Tax uncertainty associated with increases in investment and real GDP
    \ei
  \end{block}

  \pause\begin{block}{Consequences during the Great Recession}
    \bi
    \item Responsible for a 1\% to 3\% decrease in real GDP
    \item Decreased consumption by about 1\% of real GDP
    \item Decreased investment by about 1\% of real GDP
    \ei
  \end{block}
}

\frame
{
  \ft{Constant Gain Learning}
  \begin{block}{Constant gain learning mechanism}
    \bi
    \item Every period, run a least-squares regression for each fiscal policy variable, using data from previous periods.
    \item Weighted least squares - more recent observations have more weight.
    \item Regression predicted value serves as expected fiscal policy.
    \item Root (weighted) mean squared error serves as \textit{fiscal policy uncertainty}.
    \ei
  \end{block}

 \pause\begin{block}{Ideal situations for constant gain learning}
    \bi
    \item Precedence of structural changes
    \item No a-priori knowledge on menu or evolution of structural changes and probability distributions
    \item Forecasting rule, but no knowledge of parameter values, or the structure of the whole economy.
    \ei
  \end{block}

}

\frame
{
  \ft{Fiscal Policy Regressions}
  \begin{block}{Four regressions}
  \textbf{Fiscal policy variables:} $f_{t} = [g_t~ r_t~ n_t~ b_t]$ \\ 
  Govt Spending ($g_t$), Tax Revenue ($r_t$),\\
  Net Transfers ($n_t$), Government Debt / GDP ($b_t$) \\ [0.5pc]

  \textbf{Regression equation:}\\
  $f_{i,t} = \alpha_{t,0} + \alpha_{t,f}' f_{t-1} + \alpha_{t,y} y_{t} + \alpha_{t,c} c_t + \alpha_{I,t} I_t + \alpha_{t,u} u_{t} + \epsilon_t$
  \end{block}

  \begin{block}{Empirical Model for Fiscal Policy Behavior}
  Each fiscal policy variable ($f_{i,t}$) responds to:
  \bi
  \item Lag of all fiscal policy variables ($f_{t-1}$).
  \item Above includes lag of government debt ($b_{t-1}$).
  \item Macro outcomes: real GDP ($y_t$), consumption ($c_t$), investment ($I_t$), and unemployment ($u_t$).
  \item All quantities real, per capita, ratio of past real GDP.
  \ei
  \end{block}
}

\begin{comment}
\frame
{
  \ft{Least-Squares Learning}
  \begin{block}{Understanding Fiscal Policy}
    \bdm \hat{\alpha}_t = \left(\sum_{\tau=0}^{t} w_{\tau} X_{\tau} X_{\tau}' \right)^{-1} \left(\sum_{\tau=0}^{t} w_{\tau} X_{\tau}' f_{i,\tau} \right) \edm
    \bi
    \item Time $t$ expected fiscal action: $E_t^* f_{i,t} = X_t' \hat{\alpha}_{t-1}$
    \item Information set includes \textit{past fiscal behavior} and \textit{current macro conditions}.
    \item Unexplained policy: $\hat{\epsilon}_t = f_{i,t} - X_t' \hat{\alpha}_{t-1}$
    \ei
  \end{block}

  \begin{block}{Constant Gain Learning}
    \bi
    \item Weight on $t-\tau$ observation: $\omega_\tau = (1-\gamma) \gamma^{\tau}$.
    \item \textbf{Learning gain}, $\gamma \in (0,1)$, is constant weight assigned to most recent observation.
    \item $\gamma \approx 0.02$ (Milani (2008), Slobodyan and Wouters (2008)).
    \ei
  \end{block}
}


\frame
{
  \ft{Instrumental Variable Learning} 
  \begin{block}{Endogeneity Problem}
  \bi 
  \item Macro outcomes (real GDP, consumption, investment, and unemployment) are likely endogenous.
  \item Use instruments: lags of macro outcomes and fiscal variables
  \item Two-stage least squares - using constant gain weighting procedure above.
  \ei
  \end{block}

  \begin{block}{Fiscal Uncertainty}
  Unexplained fiscal policy: $\epsilon_{i,t} = f_{i,t} - \hat{\alpha}_{i,t-1}^{IV'} X_t$\\
  Fiscal Uncertainty given by Root (weighted) mean squared error:\\
    \bdm m_{i,t}^{IV} = \sqrt{ (1-\gamma) \ds \sum_{\tau=1}^{t} \gamma^{\tau} \epsilon_{i,t}^2} \edm
  \end{block}
}
\end{comment}

\frame
{
  \ft{Fiscal Policy - Actual and Predicted}
  \begin{tabular}{cc}
    \includegraphics[width=0.45\textwidth, height=0.45\textheight]{pics/pred_gov.png} & 
    \includegraphics[width=0.45\textwidth, height=0.45\textheight]{pics/pred_tax.png} \\ 
    \includegraphics[width=0.45\textwidth, height=0.45\textheight]{pics/pred_transfers.png} & 
    \includegraphics[width=0.45\textwidth, height=0.45\textheight]{pics/pred_debt.png} 
  \end{tabular}
}

\frame
{
  \ft{Fiscal Policy Uncertainty}
  \begin{tabular}{cc}
    \includegraphics[width=0.45\textwidth, height=0.45\textheight]{pics/fpu_gov.png} & 
    \includegraphics[width=0.45\textwidth, height=0.45\textheight]{pics/fpu_tax.png} \\ 
    \includegraphics[width=0.45\textwidth, height=0.45\textheight]{pics/fpu_transfers.png} & 
    \includegraphics[width=0.45\textwidth, height=0.45\textheight]{pics/fpu_debt.png} 
  \end{tabular}
}

\frame
{
  \ft{Casual Observations}
  \begin{block}{Unprecedented levels during Great Recession}
    \bi
    \item Government expenditures uncertainty: Nearly 7\% of GDP
    \item Tax uncertainty: Nearly 6\% of GDP
    \item Transfers uncertainty: Nearly 7\% of GDP
    \item Government debt uncertainty: Nearly 35\% of GDP
    \ei
  \end{block}

  \pause\begin{block}{Run up for several years preceding recessions}
    \bi
    \item Early 1980s, 2001, 2007.
    \item Not the rule though (eg: declines prior to 1970s, little volatility prior to 1991)
    \ei
  \end{block}
}

\frame
{
  \ft{Fiscal Uncertainty Correlations}
\begin{scriptsize}
\begin{center}
\textbf{Pearson Correlation Coefficient} \\ \ \\
\begin{tabular}{l|cccc}
 & Gov Spending & Tax Revenue & Transfers & Government Debt \\ \hline
Gov Spending & 1.00 & - & - & - \\
Tax Revenue & 0.75 & 1.00 & - & - \\
Transfers & 0.74 & 0.78 & 1.00 & - \\
Government Debt & 0.64 & 0.65 & 0.90 & 1.00 \\ \hline
\end{tabular}
\end{center}
\end{scriptsize}
\bi
\item All highly correlated. 
\item Common (latent) factor?
\ei
}

\frame
{
  \ft{Fiscal Uncertainty Coincident Indicator}
  \begin{block}{Objective}
    \bi 
    \item Strip out the common component of fiscal uncertainty
    \item Construct a general measure of fiscal uncertainty
    \item Take care of potential multicolinearity problem
    \item Compare to Baker, Bloom, and Davis (2013) (BBD)
    \ei
  \end{block}    

  \begin{block}{Stock and Waston (1989) coincident indicator model}
  \bi
  \item Latent variable: General fiscal uncertainty
    
    \vspace*{-0.5pc} \bdm \begin{array}{l} m_t = m_0 + A \lambda_t + e_t \\ [0.2pc]
\lambda_t = b_1 \lambda_{t-1} + b_2 \lambda_{t-2} + \upsilon_t\\ [0.2pc]
e_t = C e_{t-1} + \eta_t \end{array} \edm
     \vspace*{-0.5pc} \bi
     \item $m_t$: 4x1 vector of fiscal uncertainty variables
     \item $\lambda_t$: general fiscal uncertainty
     \item $m_0 + e_t$: idiosyncratic component of fiscal uncertainty.
     \ei
   \ei
   \end{block}
}

\frame
{
  \ft{Coincident Indicator: General Fiscal Uncertainty}
  \begin{center}
    \includegraphics[width=0.8\textwidth]{pics/fpucoin.png}
  \end{center}
}

\begin{comment}
\frame
{
  \ft{Idiosyncratic Fiscal Uncertainty}
  \begin{tabular}{cc}
    \includegraphics[width=0.45\textwidth, height=0.45\textheight]{pics/fpucoin_gov.png} & 
    \includegraphics[width=0.45\textwidth, height=0.45\textheight]{pics/fpucoin_tax.png} \\ 
    \includegraphics[width=0.45\textwidth, height=0.45\textheight]{pics/fpucoin_transfers.png} & 
    \includegraphics[width=0.45\textwidth, height=0.45\textheight]{pics/fpucoin_debt.png} 
  \end{tabular}
}
\end{comment}

\frame
{
  \ft{Fiscal Uncertainty Correlations Redux}
\begin{scriptsize}
\begin{center}
\textbf{Idiosyncratic Fiscal Uncertainty - Pearson Correlations}\\ \ \\
\begin{tabular}{l|cccc}
 & Gov Spending & Tax Revenue & Transfers & Government Debt \\ \hline
Gov Spending & 1.00 & - & - & - \\
Tax Revenue & 0.40 & 1.00 & - & - \\
Transfers & -0.17 & -0.23 & 1.00 & - \\
Government Debt & -0.21 & -0.32 & -0.18 & 1.00 \\ \hline
\end{tabular}
\end{center}
\end{scriptsize}
\ \\ \ \\

\begin{scriptsize}
\begin{center}
\textbf{Correlation of RMSE with Coincident Index}\\ \ \\
\begin{tabular}{l|cccc}
 & Gov Spending & Tax Revenue & Transfers & Government Debt \\ \hline
Coincident Index~ & 0.75 & 0.78 & 0.99 & 0.91 \\ \hline
\end{tabular}
\end{center}
\end{scriptsize}
}

\frame
{
  \ft{Relationship with Baker et. al. (2013)}
\begin{footnotesize}
\hspace*{-2pc}\begin{tabular}{ccc}
\includegraphics[scale=0.22]{./results/pics0.01/fpuindex.png} & \includegraphics[scale=0.22]{./results/pics0.02/fpuindex.png} & \includegraphics[scale=0.22]{./results/pics0.04/fpuindex.png} \\
Learning Gain = 0.01 & Learning Gain = 0.02 & Learning Gain = 0.04  \\
Correlation = 0.27 & Correlation = 0.17 & Correlation = 0.06 \\
\end{tabular}
\bi
\item Close match post-2000
\item Higher correlation with more empirically plausible learning gains
\item BBD - Headline news is likely endogenous
\item BBD - Tax policy expiration is forward looking
\item BBD is a general economic policy uncertainty index
\ei
\end{footnotesize}
}

\frame
{
  \ft{Autoregressive Distributed Lag Model}
  %\begin{footnotesize}
  \begin{block}{Dependent Variables: Macroeconomic Outcomes}
    \vspace*{-0.5pc}\begin{columns}
      \column{0.3\textwidth}
      \bi \item Real GDP \item Consumption  \ei
      \column{0.3\textwidth}
      \bi \item Investment \item Inflation \ei 
      \column{0.3\textwidth}
      \bi \item Employment \item Unemployment \ei
      \column{0.05\textwidth}~
    \end{columns}
  \end{block}

  \begin{block}{Explanatory Vars: Common and Idiosyncratic Fiscal Uncertainty}
      \vspace*{-0.5pc}\begin{columns}[t]
      \column{0.4\textwidth}
      \bi \item Government Exp  \item Tax Receipts  \item Transfer Payments  \ei
      \column{0.55\textwidth}
      \bi \item Government Debt  \item Coincident Index \ei ~~~(First lag to avoid endogeneity)
      \column{0.05\textwidth}~
    \end{columns}
  \end{block}

  \begin{block}{Controls}
    \bi 
    \item Lags of all the dependent variables in every model.
    \item Lags of all the fiscal policy variables
    \ei
  \end{block}

  %\end{footnotesize}
}

\frame
{
  \ft{ARDL Results (Learning Gain = 0.02, Lags = 2)}

\begin{tiny}
\begin{center}
\begin{tabular}{l|S[table-format=3.2] S[table-format=3.2] S[table-format=3.2] S[table-format=3.2] S[table-format=3.2] S[table-format=3.2]}
\textbf{Fiscal Uncertainty} & \multicolumn{6}{c}{\textbf{Dependent Variables (Column Headings)}} \\
\textbf{ - Row Headings -}                 & \multicolumn{1}{p{0.4in}}{\vspace*{-1.5pc}\begin{center}~\newline Real GDP\end{center}} 
                & \multicolumn{1}{p{0.4in}}{\vspace*{-1.5pc}\begin{center}Consumption\end{center}} 
                & \multicolumn{1}{p{0.4in}}{\vspace*{-1.5pc}\begin{center}~\newline Investment\end{center}} 
                & \multicolumn{1}{p{0.4in}}{\vspace*{-1.5pc}\begin{center}Employment\end{center}}
                & \multicolumn{1}{p{0.4in}}{\vspace*{-1.5pc}\begin{center}Unemployment\end{center}} 
                & \multicolumn{1}{p{0.4in}}{\vspace*{-1.5pc}\begin{center}~\newline Inflation\end{center}} \\ [-0.5pc] \hline
 & \multicolumn{6}{c}{} \\ [-0.25pc]
 Government Exp \hl{4} & -0.04 & 0.06 & -0.06 & -0.68** \hl{4} & 0.55*** \hl{4} & 0.02 \\
 (Standard Error) \hl{4} & (0.11) & (0.07) & (0.08) & (0.28) \hl{4} & (0.13) \hl{4} & (0.25) \\ [0.2pc]
 Tax Receipts \hl{6} & 0.36*** \hl{6} & 0.07  & 0.26*** \hl{6} & 0.39 & -0.22 & 0.05 \\
 (Standard Error) \hl{6} & (0.11) \hl{6} & (0.06) & (0.09) \hl{6} & (0.28) & (0.14) & (0.15) \\ [0.2pc]
 Transfer Payments \hl{4} & -0.01 & -0.03 & 0.01 & -0.49** \hl{4} & 0.19*** \hl{4}  & 0.01 \\
 (Standard Error) \hl{4} & (0.08) & (0.04) & (0.04) & (0.23) \hl{4}  & (0.06) \hl{4} & (0.12) \\ [0.2pc]
 Government Debt \hl{5}  & 0.05 & -0.03 & 0.09 & -1.27 \hl{5} & 0.25 \hl{5} & 0.12 \\
 (Standard Error) \hl{5}  & (0.10) & (0.06) & (0.06) & (0.88) \hl{5} & (0.16) \hl{5} & (0.17)  \\ [0.2pc]
 Coincident Index \hl{3} & -0.41*** \hl{3} & -0.21*** \hl{3} & -0.19*** \hl{3} & 0.13 & -0.22* & -0.36** \hl{3} \\
 (Standard Error) \hl{3} & (0.10) \hl{3} & (0.05) \hl{3} & (0.07) \hl{3} & (0.38) & (0.14) & (0.16) \hl{3} \\ [0.2pc]
\hline
 ~ \hl{2} & \multicolumn{6}{c}{ ~ \hl{2}} \\ [-0.25pc]
Joint Wald \hl{2} &  4.02*** \hl{2} &  3.80*** \hl{2} & 2.54** \hl{2} & 3.21*** \hl{2} & 4.27*** \hl{2} &  1.29 \hl{2} \\ [0.25pc] \hline

 & \multicolumn{6}{c}{} \\ [-0.25pc]
 Adjusted R-square & 0.32 & 0.98 & 0.96 & 0.83 & 0.87 & 0.81 \\
 AIC & 466.15 & 198.35 & 257.72 & 666.99 & 398.54 & 632.69 \\
 BIC & 549.83 & 282.03 & 341.40 & 750.67 & 482.22 & 716.37 \\ \hline

\end{tabular}

\end{center}
\end{tiny}
\begin{scriptsize}
\only<1>{~ \\  ~ \\ }
\only<2>{\textcolor{BrickRed}{\textbf{1. Fiscal uncertainty influences everything but inflation}}}
\only<3>{\textcolor{BrickRed}{\textbf{2. Common fiscal uncertainty dampens aggregate demand}}}
\only<4>{\textcolor{BrickRed}{\textbf{3. Transfers and Spending uncertainty drags on employment}}}
\only<5>{\textcolor{BrickRed}{\textbf{4. Debt uncertainty drags on employment (significant in most other specifications)}}}
\only<6>{\textcolor{BrickRed}{\textbf{5. Tax uncertainty (mostly unexpectedly low) boosts investment and real GDP}}}
\end{scriptsize}
}

\frame
{
  \ft{Impact from an Historical Buildup}
\begin{scriptsize}
\begin{center}
\textbf{Magnitude of Extreme Change in Coincident Fiscal Uncertainty}\\
\textbf{(Learning Gain = 0.02)} \\
~ \\
\begin{tabular}{l|S[table-format=3.2] S[table-format=3.2] S[table-format=3.2] S[table-format=3.2]} \hline
\multicolumn{3}{l}{Largest Value Coincident Fiscal Uncertainty = 4.77} & \multicolumn{1}{r}{Date: 2009 Quarter 2} & \\
\multicolumn{3}{l}{Smallest Value in Decade Preceding = -0.34} & \multicolumn{1}{r}{Date: 2005 Quarter 4} & \\ \hline
\end{tabular} \\
~ \\
~ \\
\textbf{Estimated Impact - ARDL(2)} \\
~ \\
\begin{tabular}{l|S[table-format=3.2] S[table-format=3.2] S[table-format=3.2] S[table-format=3.2]}
 \multicolumn{1}{p{0.85in}}{Variable} 
                & \multicolumn{1}{p{0.85in}}{\vspace*{-1.5pc}\begin{center}Impact\end{center}} 
                & \multicolumn{1}{p{0.85in}}{\vspace*{-1.5pc}\begin{center}95\% Lower Bound\end{center}} 
                & \multicolumn{1}{p{0.85in}}{\vspace*{-1.5pc}\begin{center}95\% Upper Bound\end{center}}\\ [-0.75pc] \hline
Real GDP &  -2.07*** & -3.04 & -1.11 \\
Consumption &  -1.06*** & -1.57 & -0.54 \\
Investment &  -0.96*** & -1.64 & -0.29 \\
Employment &  0.65 & -3.15 & 4.45 \\
Unemployment & -1.14* & -2.49 & 0.21 \\
Inflation & -1.85** & -3.50 & -0.20 \\
\hline
\end{tabular}
\end{center}
\end{scriptsize}
}


\frame
{
  \ft{Conclusions}
  \begin{block}{Consequences for Fiscal Uncertainty}
    \bi
    \item lower real GDP,
    \item lower consumption,
    \item lower investment.
    \ei
  \end{block}
 
  \pause\begin{block}{Specific fiscal variables}
    \bi
    \item Government expenditures, transfers, and debt associated with labor market contractions.
    \item Tax uncertainty associated with increases in investment and real GDP
    \ei
  \end{block}

  \pause\begin{block}{Consequences during the Great Recession}
    \bi
    \item Responsible for a 1\% to 3\% decrease in real GDP
    \item Decreased consumption by about 1\% of real GDP
    \item Decreased investment by about 1\% of real GDP
    \ei
  \end{block}
}


\end{document}

