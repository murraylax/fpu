\documentclass{beamer}
\usepackage{beamerthemeshadow}
\usepackage{verbatim}
\usepackage{grffile}

\usepackage{lastpage}
\usepackage{xcolor}
\usepackage{pgf}
\usepackage{colortbl}
\usepackage{hyperref}

\newcommand{\bi}{\begin{itemize}}
\newcommand{\ei}{\end{itemize}}
\newcommand{\be}{\begin{enumerate}}
\newcommand{\ee}{\end{enumerate}}
\newcommand{\bd}{\begin{description}}
\newcommand{\ed}{\end{description}}
\newcommand{\prbf}[1]{\textbf{#1}}
\newcommand{\prit}[1]{\textit{#1}}
\newcommand{\beq}{\begin{equation}}
\newcommand{\eeq}{\end{equation}}
\newcommand{\bdm}{\begin{displaymath}}
\newcommand{\edm}{\end{displaymath}}

\newcommand{\ft}[1]{
  \frametitle{\begin{tabular}{p{4.2in}r} \textcolor{white}{#1} & \small{\insertframenumber / \inserttotalframenumber} \end{tabular}}
  \setbeamercovered{transparent=18}
}

\newcommand{\eft}[1]{
  \frametitle{\begin{tabular}{p{4in}r} \textcolor{white}{#1} & \small{\hyperlink{f:questions}{\beamergotobutton{GO BACK}}} \end{tabular}}
  \setbeamercovered{transparent=18}
}

\newcommand{\stepinv}{\setbeamercovered{invisible}}
\newcommand{\stopinv}{\setbeamercovered{transparent=18}}
\newcommand{\uncoverinv}[1]
{
  \setbeamercovered{invisible}
  \uncover<+->{#1}
  \setbeamercovered{transparent=18}
}
\newcommand{\ans}[1]{\textcolor{blue}{#1}}
\newcommand{\ansinv}[1]
{
  \setbeamercovered{invisible}
  \uncover<+->{\textcolor{blue}{#1}}
  \setbeamercovered{transparent=18}
}
\newcommand{\setinv}{\setbeamercovered{invisible}}
\newcommand{\setvis}{\setbeamercovered{transparent=18}}
\newcommand{\centerpic}[2]
{
  \begin{center}
  \includegraphics[#1]{#2}
  \end{center}
}
\newcommand{\h}[1]{\hat{#1}}
\newcommand{\ds}{\displaystyle}

%\definecolor{light}{rgb}{1.0,0.33,0.33}
\definecolor{light}{rgb}{1.0,0.5,0.5}
\newcommand{\hl}[1]{\alt<#1>{\rowcolor{light}\hspace*{-2.1pt}} {\hspace*{-2.1pt}} }

\definecolor{mycolor}{rgb}{0.6,0.0,0.0}
\usecolortheme[named=mycolor]{structure}

\title[2013 Midwest Applied Time Series Conference]{Expected Versus Unexpected Fiscal Policy Multipliers}
\author[James Murray, University of Wisconsin - La Crosse]{
James Murray\\
Department of Economics\\
University of Wisconsin - La Crosse
}
\date{Friday, June 7, 2013}

\begin{document}

\begin{comment}
Look up John Nunley's message about fiscal policy using SVARs versus DSGE - it involves fiscal policy rules

\end{comment}

\frame{\titlepage \setcounter{framenumber}{0}}

\section{}
\subsection{Purpose}
\frame
{
  \ft{Purpose}
  \bi
  \item Estimate the impact of predicted versus unexpected fiscal policy.
  \item Contribute to an unsettled fiscal multiplier literature using SVARs.
    \bi
    \item Size of the multiplier, statistical significance, impact on spending components
    \item Identification strategy
    \item Hebous (\textit{Journal of Economic Surveys}, 2007) 
    \ei
  \item Least-squares adaptive expectations.
    \bi
    \item Surveys: Evans and Honkapohja (2008, 2010) 
    \item Fiscal multipliers with adaptive expectations: Mitra, Evans and Honkapohja (2012).
    \ei
  \ei
}
\subsection{Outline}
\frame
{
  \ft{Outline}
  \be
  \item Baseline model (SVAR): estimate the impact of fiscal policy on macro outcomes.
  \item Fiscal policy uncertainty
    \bi
    \item Expectations framework
    \item Decompose actual fiscal policy into expected and unexpected components.
    \ei
  \item Extended model (SVAR): estimate the impact on macro outcomes for
    \bi
    \item Expectations for fiscal policy (predicted value)
    \item Unexpected fiscal policy (residual)
    \ei
  \ee
}

\section{Baseline Model}
\subsection{Structural Vector Autoregression}
\frame
{
  \ft{Baseline Structural VAR}
  \bi
  \item Baseline model:
    \bdm A_0 x_t = \alpha_0 + \alpha_1 t + A(L) x_t + z_t, \edm
  \item Includes a constant and a linear trend.
  \item Endogenous vector:
    \bdm x_t = \left[ \begin{array}{c} y_t \\ c_t \\ i_t \\ u_t \\ t_t \\ g_t \end{array} \right] 
         = \left[ \begin{array}{c} \mbox{Real GDP per capita} \\ \mbox{Consumption per capita} \\ \mbox{Investment per capita} \\ 
             \mbox{Unemployment Rate} \\ \mbox{Taxes net of transfers per capita} \\ \mbox{Government consumption per capita} \end{array} \right] \edm
  \item $A_0$ captures contemporaneous causal relationships
    \bi
    \item Identified with zero-restrictions on contemporaneous relationships.
    \ei
  \ei
}

\subsection{Identification Strategy}
\begin{comment}
\frame
{
  \ft{Identification Strategy}
    \bi
    \item Implementation lag for government spending: $g_t$ does not contemporaneously respond to anything.
    \item Taxes do contemporaneously respond to real GDP, unemployment, government spending decisions.
    \item Taxes do not contemporaneously respond to consumption or investment.
    \item Taxes directly affect consumption and investment decisions.
    \item Taxes do not directly affect real GDP, only indirectly through its components.
    \item Real GDP contemporaneously determined by its components: $c_t$, $i_t$, and $g_t$.
    \item Labor market frictions prevent unemployment from contemporaneously responding to anything.
    \ei
}

\frame
{
  \ft{Identification Strategy}
These restrictions leads to the following structure:
\begin{footnotesize}
\bdm \label{eq:svar} \left[ \begin{array}{cccccc} 
     1 & a_{y,c} & a_{y,i} & a_{y,u} & 0 & a_{y,g} \\
     0 & 1 & 0 & a_{c,u} & a_{c,t} & a_{g,t} \\ 
     0 & 0 & 1 & a_{i,u} & a_{i,t} & a_{g,t} \\
     0 & 0 & 0 & 1 & 0 & 0 \\
     a_{t,y} & 0 & 0 & a_{t,u} & 1 & a_{t,g} \\
     0 & 0 & 0 & 0 & 0 & 1 \\ \end{array} \right]~ 
     \left[ \begin{array}{c} y_t \\ c_t \\ i_t \\ u_t \\ t_t \\ g_t \end{array} \right]  = 
     A(L) \left[ \begin{array}{c} y_{t-1} \\ c_{t-1} \\ i_{t-1} \\ u_{t-1} \\ t_{t-1} \\ g_{t-1} \end{array} \right] +
     \left[ \begin{array}{c} z_{y,t} \\ z_{c,t} \\ z_{i,t} \\ z_{u,t} \\ z_{t,t} \\ z_{g,t} \end{array} \right],
\edm
\end{footnotesize}
\bi
\item $A(L)$: first-order distributed lag.
\item $z_{k,t}$: independently and identically distributed shocks.
\item Measure fiscal policy impacts: impulse responses functions of macro outcomes to innovations to taxes ($z_{t,t}$) and government spending ($z_{g,t}$).
\ei
}
\end{comment}

\frame
{
  \ft{Identification Strategy}
    \bi
    \item Implementation lag for government spending: $g_t$ does not contemporaneously respond to anything.
    \item Taxes contemporaneously respond to everything: real GDP, consumption, investment, unemployment, government expenditures.
    \item Consumption contemporaneously responds to taxes, government expenditures, real GDP, and unemployment.
    \item Investment contemporaneously responds to taxes, government expenditures, and real GDP.
    \item Real GDP contemporaneously determined by its components: $c_t$, $i_t$, and $g_t$.
    \item Labor market frictions prevent unemployment from contemporaneously responding to anything.
    \ei
}

\frame
{
  \ft{Identification Strategy}
These restrictions leads to the following structure:
\begin{footnotesize}
\bdm \label{eq:svar} \left[ \begin{array}{cccccc} 
     1 & a_{y,c} & a_{y,i} & 0 & 0 & a_{y,g} \\
     a_{c,y} & 1 & 0 & a_{c,u} & a_{c,t} & a_{c,g} \\ 
     a_{i,y} & 0 & 1 & 0 & a_{i,t} & a_{i,g} \\
     0 & 0 & 0 & 1 & 0 & 0 \\
     a_{t,y} & a_{t,c} & a_{t,i} & a_{t,u} & 1 & a_{t,g} \\
     0 & 0 & 0 & 0 & 0 & 1 \\ \end{array} \right]~ 
     \left[ \begin{array}{c} y_t \\ c_t \\ i_t \\ u_t \\ t_t \\ g_t \end{array} \right]  = 
     A(L) \left[ \begin{array}{c} y_{t-1} \\ c_{t-1} \\ i_{t-1} \\ u_{t-1} \\ t_{t-1} \\ g_{t-1} \end{array} \right] +
     \left[ \begin{array}{c} z_{y,t} \\ z_{c,t} \\ z_{i,t} \\ z_{u,t} \\ z_{t,t} \\ z_{g,t} \end{array} \right],
\edm
\end{footnotesize}
\bi
\item $A(L)$: first-order distributed lag.
\item $z_{k,t}$: independently and identically distributed shocks to endogenous variables.
\item Fiscal policy responses: IRFs of macro outcomes to taxes ($z_{t,t}$) and government expenditures ($z_{g,t}$).
\ei
}

\subsection{Impulse Response Functions}

\frame
{
  \ft{Fiscal Policy Responses to Own Shocks}
  \begin{tabular}{p{2in}p{2in}}
    \includegraphics[width=2in, height=2in]{pics/irf_sh_GovExpenditures_var_GovExpenditures.png} & \includegraphics[width=2in, height=2in]{pics/irf_sh_Taxes_var_Taxes.png} \\
    \begin{center}Government expenditures shock is very persistent\end{center} & \begin{center}Taxes? Problem with Identification Scheme?\end{center} \\
  \end{tabular}
}

\frame
{
  \ft{Impulse Response: Shock to Government Spending}
  \begin{tabular}{cc}
    \includegraphics[width=2in, height=1.5in]{pics/irf_sh_GovExpenditures_var_Consumption.png} & \includegraphics[width=2in, height=1.5in]{pics/irf_sh_GovExpenditures_var_Investment.png} \\
    \includegraphics[width=2in, height=1.5in]{pics/irf_sh_GovExpenditures_var_GDP.png} & \includegraphics[width=2in, height=1.5in]{pics/irf_sh_GovExpenditures_var_Unemployment.png} 
  \end{tabular}
}

\frame
{
  \ft{Impulse Response: Shock to Taxes}
  \begin{tabular}{cc}
    \includegraphics[width=2in, height=1.5in]{pics/irf_sh_Taxes_var_Consumption.png} & \includegraphics[width=2in, height=1.5in]{pics/irf_sh_Taxes_var_Investment.png} \\
    \includegraphics[width=2in, height=1.5in]{pics/irf_sh_Taxes_var_GDP.png} & \includegraphics[width=2in, height=1.5in]{pics/irf_sh_Taxes_var_Unemployment.png} 
  \end{tabular}
}

\frame
{
  \ft{Fiscal Policy Multiplier}
  Cumulative multiplier for each period after the shock.
  \bdm \mbox{Multiplier}_T~ \equiv~ m_T~ = ~\frac{\displaystyle \sum_{\tau=0}^{T} y_{\tau}^{resp}}{\displaystyle \sum_{\tau=0}^{T} f_{\tau}^{resp}} \edm
  \bi
  \item $y_{T}^{resp}$ is the response of real GDP, $T$ periods after the initial shock.
  \item $f_{T}^{resp}$ is the response of the fiscal policy variable, $g_T$ and $t_T$ in turn.
  \ei
}

\frame
{
  \ft{Fiscal Policy Multipliers}
  \begin{tabular}{p{2in}p{2in}}
    \includegraphics[width=2in, height=2in]{pics/Mult_GovExpenditures.png} & \includegraphics[width=2in, height=2in]{pics/Mult_Taxes.png} \\
    \begin{center}Government consumption multiplier is very close to zero\end{center} & \begin{center}Tax multiplier has the opposite sign than expected.\end{center} \\ 
  \end{tabular}
}

\section{Fiscal Policy Uncertainty}
\subsection{Boundedly Rational Expectations}

\frame
{
  \ft{Knowledge of Fiscal Policy}
  \bi
  \item Market participants' set of knowledge at any point in the sample period, is not equal to my estimated SVAR from June 2013.
  \item Market participants are less informed:
    \bi
    \item Knowledge of the nature of fiscal policy at time $t$ depends on data \textit{prior} to $t$. 
    \ei
  \item Might be more informed:
    \bi
    \item Structural change not captured by my SVAR, agents at time $t$ have a better perception of their structure, using data near $t$.
    \item Announcements or news of economic events (not in this paper though).
    \ei
  \item Might have misconceptions:
    \bi
    \item Past data may be a poor guide to upcoming policy.
    \ei
  \ei
}

\frame
{
  \ft{Fiscal Policy Uncertainty}
  \bi
  \item Exact conduct of fiscal policy decisions is unknown.
  \item Boundedly rational: agents expect tax and government spending decisions respond to:
    \bi
    \item Macroeconomic variables: Unemployment, real GDP.
    \item Fiscal variables: own lag, debt.
    \ei
  \item Agents use least-squares regression models to forecast fiscal variables.
  \item Expectations are adaptive:
    \bi
    \item Agents re-estimate a regression model every quarter, updating their information set with new observation from the previous quarter.
    \item Agents put more weight on more recent observations: Constant gain, weighted least-squares forecast.
    \ei
  \ei
}

\frame
{
  \ft{Motivation for Least-Squares Learning}
  \bi
  \item Taxes and transfers: respond contemporaneously to economic conditions.
  \item Government spending: 
    \bi
    \item Even announcements could be subject to legislative adjustments or reversals.
    \item Stimulus policies are complicated mixtures of taxes, transfers, and spending.
    \item Stimulus policies involve complicated implementation lags.
    \item Forecast is for the national entire portfolio of federal, state, and local spending.
    \ei
  \item Cognitive consistency principle (Evans and Honkapohja, 2010)
  \ei
}

\subsection{Fiscal Policy Rules}
\frame
{
  \ft{Fiscal Policy Rules}
  \begin{block}{Fiscal policy forecasting models}
  \vspace*{-1pc}\bdm \begin{array}{c} \label{eq:fiscalrule}
  g_t = \alpha_0 + \rho_g g_{t-1} + \alpha_y(L) y_t + \alpha_u(L) u_t + \alpha_d d_{t-1} + \epsilon_{g,t} \\ [0.5pc] 
  t_t = \beta_0 + \rho_t t_{t-1} + \beta_u(L) y_t + \beta_u(L) u_t + \beta_d d_{t-1} + \epsilon_{t,t},
  \end{array} \edm
  \end{block}
  \begin{block}{Notation}
  \begin{columns}[t]
    \column{1.7in}
    \bi
    \item $g_t$: Gov spending
    \item $t_t$: Net taxes 
    \item $y_t$: Real GDP
    \item $u_t$: Unemployment
    \item $d_t$: Government debt
    \ei
    \column{1.9in}
    \bi
    \item $\alpha_0$, $\beta_0$: constant terms
    \item $\rho_g$, $\rho_t$: persistence
    \item $\alpha_d$, $\beta_d$: response to debt
    \item $\alpha_y(L)$, $\alpha_u(L)$:\newline 2nd order distributed lag polynomials.
    \ei
    \end{columns}
  \end{block}
}

\frame
{
  \ft{Least-Squares Learning}
  \begin{scriptsize}
  \begin{block}{OLS Regression}
    Time $t$ estimates of the regression coefficients:
    \vspace*{-1pc}\bdm \hat{\Phi}_t = \left(\sum_{\tau=0}^{t} X_{\tau} X_{\tau}' \right)^{-1} \left(\sum_{\tau=0}^{t} X_{\tau}' f_{\tau} \right) \edm
    \vspace*{-1pc}\bi
    \item $f_{\tau} \in \{g_\tau, t_\tau\}$ is fiscal policy policy variable.
    \item $X_{\tau}$ is the vector of explanatory variables in the regression equation (per-determined at $\tau$).
    \item Predicted fiscal policy: $\hat{f}_t = X_t' \hat{\Phi}_{t-1}$
    \item Unexpected policy: $\hat{\epsilon}_{f,t} = f_t - X_t' \hat{\Phi}_{t-1}$
    \ei
  \end{block}

  \begin{block}{Recursive Formulation}
  The OLS regression coefficients can be rewritten as:
     \vspace*{-0.5pc}\bdm \begin{array}{c}
      R_t = R_{t-1} + \gamma_{t} (X_t X_t' - R_{t-1}), \\ [0.5pc]
      \hat{\Phi}_t = \Phi_{t-1} + \gamma_{t} R_t^{-1} X_t (f_t - X_t' \hat{\Phi}_{t-1})    \vspace*{-0.5pc}
    \end{array}\edm
   where $\gamma_{t} = 1/t$ is the \textbf{learning gain}.
  \end{block}
  \end{scriptsize}
}

\subsection{Constant Gain Learning}
\frame
{
  \ft{Constant Gain Learning}
  \begin{block}{Constant gain framework}
    \bi
    \item Replace $\gamma_t$ with a constant, $\gamma \in (0,1)$.
    \item Weighted least squares - more recent observations have more weight.
    \ei
  \end{block}

  \begin{block}{Ideal situations for constant gain learning}
    \bi
    \item Precedence of structural changes.
    \item No a-priori knowledge on menu of structural changes and probability distributions.
    \item Reasonable that learning dynamics should not disappear with time.
    \ei
  \end{block}
}

\frame
{
  \ft{Constant-Gain Learning}
  \begin{footnotesize}
  \begin{block}{Constant Gain Recursive Formulation}
    \bdm \begin{array}{c}
      R_t = R_{t-1} + \gamma (X_t X_t' - R_{t-1}), \\ [0.5pc]
      \hat{\Phi}_t = \Phi_{t-1} + \gamma R_t^{-1} X_t (f_t - X_t' \hat{\Phi}_{t-1}) 
    \end{array}\edm
    \bi
    \item Learning gain, $\gamma \in (0,1)$, is constant, related to the weight assigned to most recent observation.
    \item Typical estimates for $\gamma \sim 0.02$ (Milani (2008), Slobodyan and Wouters (2008)).
    \ei
  \end{block}

  \begin{block}{Standard Formulation}
    \bdm \hat{\Phi}_t = \left( (1-\gamma)  \sum_{\tau=1}^{t} \gamma^{\tau} X_{t-\tau} X_{t-\tau}' \right)^{-1}  \left( (1-\gamma)  \sum_{\tau=1}^{t} \gamma^{\tau} X_{t-\tau}  f_{t-\tau} \right). \edm
    Weight on $t-\tau$ observation declines geometrically with $\tau$: $\omega_\tau = (1-\gamma) \gamma^{\tau}$.
  \end{block}
  \end{footnotesize}
}

\subsection{Evolution of Expectations}
\frame
{
  \ft{Dependent variable: Government Consumption}
  \begin{center}\textbf{Evolution of Coefficients}\end{center}
  \vspace*{1pc}
  \begin{tabular}{cc}
    \includegraphics[width=2in, height=1.6in]{pics/coef_govcons_gdp.png} & \includegraphics[width=2in, height=1.6in]{pics/coef_govcons_debt.png} \\
    Real GDP & Government Debt \\
    \scriptsize{(Expect: Negative)} & \scriptsize{(Expect: Negative)} \\
  \end{tabular}
}

\frame
{
  \ft{Dependent variable: Government Consumption}
  \begin{center}\textbf{Evolution of Coefficients}\end{center}
  \vspace*{1pc}
  \begin{tabular}{cc}
     \includegraphics[width=2in, height=1.6in]{pics/coef_govcons_unemployment.png} & \includegraphics[width=2in, height=1.6in]{pics/coef_govcons_lag.png} \\
    Unemployment & Own Lag \\ 
    \scriptsize{(Expect: Positive)} & \scriptsize{(Expect: Between 0 and 1)} \\
  \end{tabular}
}

\frame
{
  \ft{Dependent variable: Government Consumption}
  \begin{center}\textbf{Predicted and Unexpected Government Consumption}\end{center}
  \vspace*{1pc}
  \begin{tabular}{cc}
     \includegraphics[width=2in, height=1.6in]{pics/govcons.png} & \includegraphics[width=2in, height=1.6in]{pics/unexp_govcons.png} \\
  \end{tabular}
}

\frame
{
  \ft{Dependent variable: Taxes}
  \begin{center}\textbf{Evolution of Coefficients}\end{center}
  \vspace*{1pc}
  \begin{tabular}{cc}
    \includegraphics[width=2in, height=1.6in]{pics/coef_tax_gdp.png} & \includegraphics[width=2in, height=1.6in]{pics/coef_tax_debt.png} \\
    Real GDP & Government Debt \\
    \scriptsize{(Expect: Positive)} & \scriptsize{(Expect: Positive)} \\
  \end{tabular}
}

\frame
{
  \ft{Dependent variable: Taxes}
  \begin{center}\textbf{Evolution of Coefficients}\end{center}
  \vspace*{1pc}
  \begin{tabular}{cc}
     \includegraphics[width=2in, height=1.6in]{pics/coef_tax_unemployment.png} & \includegraphics[width=2in, height=1.6in]{pics/coef_tax_lag.png} \\
    Unemployment & Own Lag \\ 
    \scriptsize{(Expect: Negative)} & \scriptsize{(Expect: Between 0 and 1)} \\
  \end{tabular}
}

\frame
{
  \ft{Dependent variable: Taxes}
  \begin{center}\textbf{Predicted and Unexpected Taxes Net of Transfers}\end{center}
  \vspace*{1pc}
  \begin{tabular}{cc}
     \includegraphics[width=2in, height=1.6in]{pics/taxrev.png} & \includegraphics[width=2in, height=1.6in]{pics/unexp_tax.png} \\
  \end{tabular}
}


\section{Model with Expected and Unexpected Fiscal Policy}
\subsection{Structural VAR}
\frame
{
  \ft{Extended Structural VAR}
  \bi
  \item Model:
    \bdm A_0 x_t = A(L) x_t + z_t, \edm
  \item Endogenous vector:
    \bdm x_t = \left[ \begin{array}{c} y_t \\ c_t \\ i_t \\ u_t \\ \hat{\epsilon}_{t,t} \\ \hat{\epsilon}_{g,t} \\ \hat{t}_t \\ \hat{g}_t \end{array} \right] 
         = \left[ \begin{array}{c} \mbox{Real GDP} \\ \mbox{Consumption} \\ \mbox{Investment} \\ 
             \mbox{Unemployment Rate} \\ \mbox{Unexpected Net Taxes} \\ \mbox{Unexpected Government Spending} \\
             \mbox{Expected Net Taxes} \\ \mbox{Expected Government Spending}\end{array} \right] \edm
  \ei
}

\subsection{Identification Strategy}
\frame
{
  \ft{Identification Restrictions}
  \bi
  \item Similar to above: treat unexpected fiscal policies in the same manner as fiscal policy in the baseline.
  \item Expectations of fiscal policy in the current period are predetermined.
    \bi
    \item Nothing contemporaneously affects expected fiscal policy.
    \item Expected fiscal policy may contemporaneously affect consumption, investment, unemployment, unexpected fiscal policy.
    \ei
  \ei
}

\frame
{
  \ft{Identification Strategy}
Structural VAR with identification restrictions:
\begin{tiny}
\bdm \label{eq:svar} \left[ \begin{array}{cccccccc} 
     1 & a_{y,c} & a_{y,i} & 0 & 0 & a_{y,g} & 0 & a_{y,g}^e \\
     0 & 1 & 0 & a_{c,u} & a_{c,t} & a_{c,g} & a_{c,t}^e & a_{c,g}^e \\ 
     a_{i,y} & 0 & 1 & 0 & a_{i,t} & a_{i,g} & a_{i,t}^e & a_{i,g}^e \\
     0 & 0 & 0 & 1 & 0 & 0 & a_{u,t}^e & a_{u,g}^e \\
     a_{t,y} & a_{t,c} & a_{t,i} & a_{t,u} & 1 & a_{t,g} & a_{t,t}^e & a_{t,g}^e \\
     0 & 0 & 0 & 0 & 0 & 1 & a_{g,t}^e & a_{g,g}^e \\ 
     0 & 0 & 0 & 0 & 0 & 0 & 1 & 0 \\
     0 & 0 & 0 & 0 & 0 & 0 & 0 & 1 \\ \end{array} \right]~ 
     \left[ \begin{array}{c} y_t \\ c_t \\ i_t \\ u_t \\ \hat{\epsilon}_{t,t} \\ \hat{\epsilon}_{g,t} \\ \hat{t}_t \\ \hat{g}_t \end{array} \right]  = 
     A(L) \left[ \begin{array}{c} y_{t-1} \\ c_{t-1} \\ i_{t-1} \\ u_{t-1} \\ \hat{\epsilon}_{t,t-1} \\ \hat{\epsilon}_{g,t-1} \\ \hat{t}_{t-1} \\ \hat{g}_t  \end{array} \right] +
     \left[ \begin{array}{c} z_{y,t} \\ z_{c,t} \\ z_{i,t} \\ z_{u,t} \\ z_{t,t}^\epsilon \\ z_{g,t}^\epsilon \\ z_{\hat{t},t} \\ z_{\hat{g},t} \end{array} \right]
\edm
\end{tiny}
Impulse response functions to measure the impact of fiscal policy: 
  \bi
  \item Expected fiscal policy: innovations to $z_{\hat{t},t}$ and $z_{\hat{g},t}$.
  \item Unexpected fiscal policy: innovations to $z_{t,t}^\epsilon$ and $z_{g,t}^\epsilon$
  \ei
}


\subsection{Impulse Response Functions}
\frame
{
  \ft{Fiscal Policy Responses to Own Shocks}
  \begin{tabular}{cc}
    \textbf{Expected Gov Expenditures} & \textbf{Unexpected Gov Expenditures} \\
    \includegraphics[width=2in, height=1.1in]{pics/irf_sh_Predicted\space Government\space Expenditures_var_Predicted\space Government\space Expenditures.png} & \includegraphics[width=2in, height=1.1in]{pics/irf_sh_Unexpected\space Government\space Expenditures_var_Unexpected\space Government\space Expenditures.png} \\
    \textbf{Expected Taxes} & \textbf{Unexpected Taxes} \\
    \includegraphics[width=2in, height=1.1in]{pics/irf_sh_Predicted\space Taxes_var_Predicted\space Taxes.png} & \includegraphics[width=2in, height=1.1in]{pics/irf_sh_Unexpected\space Taxes_var_Unexpected\space Taxes.png} 
  \end{tabular}
}

\frame
{
  \ft{Impulse Response: Government Expenditures}
  \begin{tabular}{cc}
    \textbf{Expected Gov Expenditures} & \textbf{Unexpected Gov Expenditures} \\
    \includegraphics[width=2in, height=1.3in]{pics/irf_sh_Predicted\space Government\space Expenditures_var_Consumption.png} & \includegraphics[width=2in, height=1.3in]{pics/irf_sh_Unexpected\space Government\space Expenditures_var_Consumption.png} \\
    \includegraphics[width=2in, height=1.3in]{pics/irf_sh_Predicted\space Government\space Expenditures_var_GDP.png} & \includegraphics[width=2in, height=1.3in]{pics/irf_sh_Unexpected\space Government\space Expenditures_var_GDP.png} 
  \end{tabular}
}

\frame
{
  \ft{Impulse Response: Net Taxes}
  \begin{tabular}{cc}
    \textbf{Expected Net Taxes} & \textbf{Unexpected Net Taxes} \\
    \includegraphics[width=2in, height=1.3in]{pics/irf_sh_Predicted\space Taxes_var_Consumption.png} & \includegraphics[width=2in, height=1.3in]{pics/irf_sh_Unexpected\space Taxes_var_Consumption.png} \\
    \includegraphics[width=2in, height=1.3in]{pics/irf_sh_Predicted\space Taxes_var_GDP.png} & \includegraphics[width=2in, height=1.3in]{pics/irf_sh_Unexpected\space Taxes_var_GDP.png} 
  \end{tabular}
}

\frame
{
  \ft{Fiscal Policy Multipliers}

  \bdm \mbox{Multiplier}_{T} = \frac{\displaystyle \sum_{\tau=0}^{T} y_{\tau}^{resp}}{\displaystyle \sum_{\tau=0}^{T} \hat{f}_{\tau}^{resp} + \hat{\epsilon}_{f,\tau}^{resp}} \edm

  \bi
  \item $y_{T}^{resp}$ is the response of real GDP, $T$ periods after the initial shock.
  \item $\hat{f}_{T}^{resp}$ is the response of the predicted fiscal policy variable, $\hat{g}_T$ and $\hat{t}_T$, in turn.
  \item $\hat{\epsilon}_{f,\tau}^{resp}$ is the response of the unexpected (residual) fiscal policy variable, $\hat{\epsilon}_{g,T}$ and $\hat{\epsilon}_{t,T}$, in turn.
  \ei
}

\frame
{
  \ft{Fiscal Policy Multipliers}
  \begin{tabular}{p{2in}p{2in}}
    \includegraphics[width=2in, height=2in]{pics/Mult_GovExpenditures_both.png} & \includegraphics[width=2in, height=2in]{pics/Mult_Taxes_both.png} \\
    \begin{center}Expected fiscal policy multiplier is larger in longer horizons\end{center} & \begin{center}Something's wrong.\end{center} \\ 
  \end{tabular}
}

\section{}
\subsection{Closing Remarks}
\frame
{
  \ft{Conclusions for Fiscal Policy}
  \bi
  \item Government spending: 
    \bi
    \item Timing of the response different for shocks to expected versus unexpected policy.
    \item Consumption responses to expected fiscal policy are statistically significant.
    \item Consumption responses to unexpected fiscal policy are not statistically significant.
    \item Expected fiscal policy multiplier is larger in longer horizons.
    \ei
  \item Taxes:
    \bi
    \item Responses to unexpected shocks similar to baseline model tax shocks.
    \item Unexpected tax shocks are muted - net taxes are an automatic stabilizer.
    \ei
  \ei
}

\frame
{
  \ft{Moving Forward}
  \bi
  \item Sign restrictions identification strategy (Mountford and Uhlig, 2009).
  \item Minimal assumptions to identify minimal shocks:
    \bi
    \item Business cycle or monetary shock: increase output, consumption, and investment; decrease unemployment.
    \item Fiscal policy shock: fiscal policy variable increases, shock is orthogonal to shock above.
    \ei
  \ei
}

\end{document}

